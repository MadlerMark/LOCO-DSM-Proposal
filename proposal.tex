\documentclass[sigplan,nonacm]{acmart}
\usepackage[normalem]{ulem}
\usepackage{wrapfig}
\usepackage{listings}
\usepackage{caption}
\usepackage{subcaption}
\title{Proposal}
\author{Mark Madler}
\begin{document}
\maketitle

\section{Design}
This work will consist of modifying an existing compiler (gcc or llvm**) to 
support LOCO as a backend for disaggregated memory. Ideally allowing programs 
written for OpenMP to work seamlessly in a disaggregated setting.
These modifications include modifying dynamic memory allocation to use special 
virtual addresses for LOCO objects and modifying load/store behavior for these 
special addresses. 
There will also be a system to determine whether a remote address is cached or not.
All cached data will be tagged with a dirty delta to allow for seamless release consistency. 
On releases data will be written back to its home-node and only modified data will be written back 
(scatter). If two nodes wish to write back the same byte-area they are in a data race 
and it is the programmer's fault. 

The first iteration of this design should include all heap memory in a LOCO hashtable and
simply perform read/write for each load/store and an insert/delete for each new/delete. 
Since all shared data for OpenMP is on the heap, there is a nice analog to insert/delete 
for each allocation or deallocation of shared data. Tentatively it seems that stack
accesses can be ignored by leaving calls to alloca alone and ensuring that "regular" 
memory addresses behave as normal.

\subsection{New and Malloc}
Calls to malloc and new need to be handled correctly in the frontend. So in clang
there should be a call to a special LOCO runtime function to allocation LOCO memory.
e.g. locomalloc. (basically just an insert at a new special memory address)

% This modification could also just be done on LOCO itself by overloading new.

\subsection{Load / Store behavior}
Loads and stores to new LOCO memory locations need to be handled differently. 
We will create new functions remoteload and remotestore which will replace
loads and stores in LLVM IR based on the virtual address of the operands.
This can be done in a special pass after the LLVM IR has been created which will maintain the 
modularity that LLVM has. There may also be a need to modify atomic loads and stores 
differently than others, but lock-guarded accesses should behave identically.

\begin{figure}
    \centering
    \begin{subfigure}{0.45\textwidth}
    \begin{lstlisting}[basicstyle={\ttfamily\scriptsize}]
    for (auto &I : instructions(F)) {
      if (auto *Load = dyn_cast<LoadInst>(&I)) {
        if (isRemoteAddress(Load->getPointerOperand())) {
          // Replace with remote_load
          IRBuilder<> Builder(Load);
          Value *RemoteLoadCall = Builder.CreateCall(
              RemoteLoadFunc, {Load->getPointerOperand()});
          Load->replaceAllUsesWith(RemoteLoadCall);
          Load->eraseFromParent();
          }
      }
    }
    \end{lstlisting}
    \end{subfigure}
    \caption{IR modification code}
    \label{fig:ir-modification}
    \end{figure}

the load store modification might look something like Figure~\ref{fig:ir-modification}



\subsection{synchronization primitives}
For this backend to work seamlessly with LOCO there may be a need to replace 
synchronization primitives with LOCO analogs. I am unsure of how these primitives might 
work with things like task synchronization and thread numbering across nodes.

% There will be {} major components in this design. They are as follows:

% \begin{itemize}
%     \item Load/store compiler modification. Need to detect these special loads and stores
%     So this might be in LLVM's SelectionDAGBuilder.cpp in "visitStore" and "visitLoad". 
%     llvm-project/llvm/lib/CodeGen/SelectionDAG/SelectionDAGBuilder.
%     \item LOCO loads/stores need to be tagged at some point. Either these objects should 
%     be at special virtual addresses by modifying malloc things. I am thinking this will be 
%     done in clang (llvm-project/clang/lib/CodeGen/CGExprCXX.cpp) what I found here should work 
%     for calls with new, note sure about malloc. Malloc might be in CGCall??
%     \item synchronization primitive modification for OpenMP.
%     \item non-cached memory will be stored as a hash map.
% \end{itemize}

\section{outstanding questions}
some outstanding questions to answer:
\begin{itemize}
    \item how does the local node know what data is local? -- hashtable lookup for local heap 
    may be somewhat costly
    \item does a system like this exist for TCP/IP? -- not to my knowledge
    \item specifics of design implementation. i.e does loco have all openMP primitives? It 
    seems that LOCO has a corresponding synchronization primitive for each of openMP's, but it 
    may be difficult to determine when to use each primitive.
    \item should GCC or LLVM be used? It looks like GCC has hooks for malloc/realloc/free which 
    would allow for easy integration. But then I am not sure where the insertion point for modifying 
    load/stores is. LLVM on the other hand is very modular and it would be easy to add a shim layer 
    to modify IR. 
\end{itemize}

\section{My TODO list / other notes on project progression}
\begin{itemize}
    \item get cmatose.c working on r320s to diagnose issue. (is it librdmacm?). cmatose
    works. The issue must be in our configuration -- but why does hardware impact this?
    \item invalid args from loco on r320s, seems to not in in hostname lookup, but maybe 
    in rdmagetaddrinfo from that hostname. Will need to check and debug that specifically.
    \item change all config scripts in Odyssey.
    \item Finish this proposal.
\end{itemize}


% \bibliographystyle{acm}
% \bibliography{relworks}
\end{document}