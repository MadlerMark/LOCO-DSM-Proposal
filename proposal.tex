\documentclass[sigplan,nonacm]{acmart}
\usepackage[normalem]{ulem}
\usepackage{wrapfig}
\usepackage{listings}
\usepackage{caption}
\usepackage{subcaption}
\title{Proposal}
\author{Mark Madler}
\begin{document}
\maketitle

\section{Design}
This work will consist of modifying an existing compiler (gcc or llvm**) to 
support LOCO as a backend for disaggregated memory. Ideally allowing programs 
written for OpenMP to work seamlessly in a disaggregated setting.
These modifications include modifying dynamic memory allocation to use special 
virtual addresses for LOCO objects and modifying load/store behavior for these 
special addresses. 
There will also be a system to determine whether a remote address is cached or not.
All cached data will be tagged with a dirty delta to allow for seamless release consistency. 
On releases data will be written back to its home-node and only modified data will be written back 
(scatter). If two nodes wish to write back the same byte-area they are in a data race 
and it is the programmer's fault. 

The first iteration of this design should include all heap memory in a LOCO hashtable and
simply perform read/write for each load/store and an insert/delete for each new/delete. 
Since all shared data for OpenMP is on the heap, there is a nice analog to insert/delete 
for each allocation or deallocation of shared data. Tentatively it seems that stack
accesses can be ignored by leaving calls to alloca alone and ensuring that "regular" 
memory addresses behave as normal.

\subsection{New and Malloc}
Calls to malloc and new need to be handled correctly in the frontend. So either in clang
or in the LLVM IR there should be a call to a special LOCO runtime function to allocation 
LOCO memory. e.g. locomalloc. (basically just an insert at a new special memory address).
Refer to this tutorial on adding a pass to LLVM IR: \url{https://www.cs.cornell.edu/%7Easampson/blog/llvm.html}

The first iteration of this design might look like just replacing every call to malloc 
with a special malloc that creates an entry in LOCO hashtable. Later this should only 
be done for shared variables. I am unsure as of now how to determine what is shared
but it seems like these things would need to be changed in the clang frontend.

\subsection{Load / Store Behavior}
Loads and stores to new LOCO memory locations need to be handled differently. 
We will create new functions remoteload and remotestore which will replace
loads and stores in LLVM IR based on the virtual address of the operands.
This can be done in a special pass after the LLVM IR has been created which will maintain the 
modularity that LLVM has. There may also be a need to modify atomic loads and stores 
differently than others, but lock-guarded accesses should behave identically.

\begin{figure}
    \centering
    \begin{subfigure}{0.45\textwidth}
    \begin{lstlisting}[basicstyle={\ttfamily\scriptsize}]
    for (auto &I : instructions(F)) {
      if (auto *Load = dyn_cast<LoadInst>(&I)) {
        if (isRemoteAddress(Load->getPointerOperand())) {
          // Replace with remote_load
          IRBuilder<> Builder(Load);
          Value *RemoteLoadCall = Builder.CreateCall(
              RemoteLoadFunc, {Load->getPointerOperand()});
          Load->replaceAllUsesWith(RemoteLoadCall);
          Load->eraseFromParent();
          }
      }
    }
    \end{lstlisting}
    \end{subfigure}
    \caption{IR modification code}
    \label{fig:ir-modification}
    \end{figure}

the load store modification might look something like Figure~\ref{fig:ir-modification}

% \subsection{Synchronization Primitives}
% For this backend to work seamlessly with LOCO there may be a need to replace 
% synchronization primitives with LOCO analogs. I am unsure of how these primitives might 
% work with things like task synchronization and thread numbering across nodes.


\subsection{Memory Consistency}
The memory consistency model in this design should be relatively simple. We aim for 
release consistency. I would assume that on each write release all local changes
are pushed to their remote locations ensuring all previous writes are visible before
and subsequent writes. If doing an acquire load it may be necessary to pull remote 
changes. In LLVM IR loads may be atomic and, if they are, they will have a tag 
denoting whether they are acquires or releases.


\subsection{Caching}
To cache data while all loads and stores are handled through this new hashtable 
mechanism we will require a scheme to save all read-accesses temporarily. Since 
we are not modifying how the stack behaves this could be where local cache is, this 
also works because it is assumed that cached data is not shared between threads.
I think the most logical thing to do is to store all hashtable accesses on the stack 
and to add them to some cache-table which will be accessed first on lookups. If there 
is a acquire load or a write release RDMA operations are required.

\subsection{OpenMP Integration and Synchronization Primitives}
LOCO supports seemingly all OpenMP synchronization primitives, below is a table of 
analogous operations:

\begin{center}
    \begin{tabular}{ |c|c| } 
     \hline
     omp-Barrier & LOCO-Barrier (either local or Global) \\ 
     \hline
     omp-Flush & write + LOCO fence\\ 
     \hline
     omp-critical & use LOCO locks \\ 
     \hline
     omp-atomic & use LOCO atomic writes \\
     \hline
     omp-ordered & use barriers? \\
     \hline
    \end{tabular}
\end{center}


\section{outstanding questions}
some outstanding questions to answer:
\begin{itemize}
    \item how does the local node know what data is local? -- hashtable lookup for local heap 
    may be somewhat costly
    \item does a system like this exist for TCP/IP? -- not to my knowledge
    \item specifics of design implementation. i.e does loco have all openMP primitives? It 
    seems that LOCO has a corresponding synchronization primitive for each of openMP's, but it 
    may be difficult to determine when to use each primitive.
    \item should GCC or LLVM be used? It looks like GCC has hooks for malloc/realloc/free which 
    would allow for easy integration. But then I am not sure where the insertion point for modifying 
    load/stores is. LLVM on the other hand is very modular and it would be easy to add a shim layer 
    to modify IR. 
\end{itemize}

\section{My TODO list / other notes on project progression}
\begin{itemize}
    \item get cmatose.c working on r320s to diagnose issue. (is it librdmacm?). cmatose
    works. The issue must be in our configuration -- but why does hardware impact this?
    \item invalid args from loco on r320s, seems to not in in hostname lookup, but errors 
    on \texttt{rdma\_connect()} as called from \texttt{lf\_peer} l.329 from \texttt{lf\_network::init()} and manager.
    It appears that there is some issue with how the endpoint is setup or in the connection 
    parameters. Maybe this issue is due to the fact that the NIC may not support IPv4? 
    It really seems like this is because of hostname issues. I think that looking up addresses
    from hostname is not working in this context. The IP address associated with the destination 
    is not even found in the dns server?? It is some address in Korea..
    \item change all config scripts in Odyssey. Changes seem to be mostly made, but 
    still there is issue where copy-run.sh stalls after building
    \item Finish this proposal.
\end{itemize}


% \bibliographystyle{acm}
% \bibliography{relworks}
\end{document}

% There will be {} major components in this design. They are as follows:
% \begin{itemize}
%     \item Load/store compiler modification. Need to detect these special loads and stores
%     So this might be in LLVM's SelectionDAGBuilder.cpp in "visitStore" and "visitLoad". 
%     llvm-project/llvm/lib/CodeGen/SelectionDAG/SelectionDAGBuilder.
%     \item LOCO loads/stores need to be tagged at some point. Either these objects should 
%     be at special virtual addresses by modifying malloc things. I am thinking this will be 
%     done in clang (llvm-project/clang/lib/CodeGen/CGExprCXX.cpp) what I found here should work 
%     for calls with new, note sure about malloc. Malloc might be in CGCall??
%     \item synchronization primitive modification for OpenMP.
%     \item non-cached memory will be stored as a hash map.
% \end{itemize}